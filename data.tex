\section{Overview}
Imagine the following scenario (inspired by the YouTube channel:[]):
Today we are going to draw a smiling ice-cream cone. Okay, we are going to first draw a curve as the top of a big scoop of ice-cream. Next, we will draw a sequence of connected U's to represent the bottom of the overflowing ice-cream. Lastly, we will draw a large upside-down triangle as the cone of our ice-cream.

We want to realize this kind of interactions with a robot, as a companion, so we need to collect a dataset that can help us to get closer to this goal. In order to study this problem, we want to collect human sketches, so the first thing we did was designing an web interface.  
The leading questions of the data collection process. Our goal is to collect a dataset so that we can learn a model that can interactively draw sketches with users. Therefore, we want to collect the drawing for a single step and a person's description of the drawing. Our design of the interface centers around some key questions: 
\begin{enumerate}
    \item \label{data_design_3} Ensure that the drawing responds to the prompt. The underlying assumption here is that the prompt itself will give us some signals in terms of where the objects in the images might be.  
    \item \label{data_design_1} From the design side, enforce annotators to breaking the sketch generation process into steps. The worst scenarios is for the annotators to   
    \item \label{data_design_2} How do we make sure that annotators are breaking the sketches they provide into reasonable steps? What we mean by reasonable here is the fact that there should be a good correspondence between some parts of the sketch and the language that is used to describe it. Although in our daily interactions, we might say something like ``we now draw this'' or ``we can do this'', but from a model learning perspective, or more so as a first step, we want there to be little ambiguity in our language and disallowing words like ``this''.     
    % {\color{red} \faIcon{question}} How to make sure that users can do not go outside the step? How to make sure that users do not miss a step? How to make sure that users do not cram two steps together? How to make sure that they remember to annotate for a step and also give the accurate descriptions? 
\end{enumerate}



% \setlist[enumerate,1]{leftmargin=12mm}
% \begin{enumerate}[itemsep=6pt,topsep=6pt]
%   \usageitem{\centering \faBook} \label{test_1} \textbf{Dictionary} is ...
%   \usageitem{\centering \large \faIcon{phone-square-alt}} \textbf{Mobile phones} are cool ...
% \end{enumerate}
% Reflects main requirement \ref{test_1}{\faBook}

Our interface has experienced 2 main versions, and the major difference between the two is that the first one asks users to draw the sketches and annotate each step in their drawings while the second version asks the users to annotate existing sketches. The turning point happens after a pilot deployment of the first version, during which we identified several problems: (1) users take too long to complete one task, and it is outside our budge to collect an ample dataset; (2) users cannot separate the entire sketch into steps consistently, and the annotations either describe more or less than what was done in a single step. In order to shorten the task time and alleviate the burden to think about how to draw certain objects, our second version uses sketches from the Quick,Draw! dataset collected by Google and asks users to provide textual annotations for each part in the sketches. The following sections will walk through each version and discuss the data collected using each version. The following sections will walk through each version and discuss how the design reflects or answers the above N criteria and what in reality happened that caused us to change the design.  

Later, we discovered that by simply using existing sketches without asking for users to draw for the prompt would significantly reduce the data collection time, and it would also allow us to put aside DQ \ref{data_design_1}. In general, if you think about it, classic collection tasks such as assigning label to images/texts or drawing segmentation box, the goal of the task is very clear, and it is easy to determine the quality of the work when you glance at it, or easy to verify. At the beginning, we found it very difficult to describe what should be drawn and what should not be drawn, or what can be written and what cannot be written. 

The general trend of the data collection process is that we try to simplify the data collection interface and reduce the number of criteria that we need to satisfy, since each introduces a factor of uncertainty.   

\section{Version 0}
Since the beginning of data collection, an important question we try to answer is how do we define a semantic unit in the sketch?  
The end goal is to achieve the kind of interaction shown in the YouTube video \textit{How To Draw A Cute Ice Cream Cone}, and in it, the instructor oftens uses sentences in the form of ``Let's draw a \texttt{X} for \texttt{Y}'', where \texttt{X} describes the geometric features of the object \texttt{Y}. For example, ``Let's draw \textit{small connected U shapes} for the \textit{bottom of the ice-cream cone}.'' Therefore, at first, we thought of decomposing the drawing process into a sequence of common geometric shapes, and the objects that they represent become the basic semantic units.      
At each step, the annotator is first asked to 
Version 0 was never deployed. I think at this stage of the data collection, we are trying to decide whether there should be a fixed set of primitive that the users could choose from, so learning the model becomes learning to parameterize, for example, the dimensions of the set of primitives. 

Functionality:
\begin{itemize}
    \item Draw the figure and the page will record the sequence 
    \item User can replay its drawing sequence. The original idea was that users will first create the drawing, and then they can replay the sequence as they annotate for each step. 
\end{itemize}

The very first test version:
In terms of the main task, I created a test version to confirm that the idea of the drawing board is sensible.

Press \textit{Record}, Draw on the board, Press \textit{Stop} when done with drawing, \textit{Submit} the drawing if one is satisfied with the quality, \textit{Play} to revisit the drawing, \textit{Cancel} to start over. 

What was the original motivation behind this functionality was that it will aid the annotators to review the drawing process and divide it into better steps. Responding to DQ \ref{data_design_1}. 
However, in this very crude version, we did not really incorporate features for either 
Responding to DQ \ref{data_design_2}, 

We begin with a very crude version, and then we decide to add features that can allow us to realize the DQ \ref{data_design_1} and \ref{data_design_2}.

The actual Version 0 has the following flow:
There is a practice board, you can try to practice drawing so that the actual drawing submit has good quality and respond to the prompt (reflecting DQ \ref{data_design_3}). Then hit \textit{Ready to Record}, again baking the sequence into the design of the website will help us to enforce collecting a dataset of steps. Another purpose is to help the annotators decide beforehand what are the necessary primitives used in the process. Why was I so fixated on the primitives, because the abstractness of the icons is what interested me the most. The entire research journey was very explorative, it sorted of started with a sense of \textit{oh, this question or aspect of how humans do things is interesting, I wish robot can do the same}. And what is that thing that I thought was interesting, it was how Rain and I were able to draw the icons and the interactions. 
The first thing you will do is select a primitive from a list, and then you will draw the step that contains the primitives. Hit \textit{Next} to move on to drawing the next primitive. There are will be a little tag at the bottom showing what is the primitive that corresponds to the step that is drawn on the board. Repeat until finished and hit \textit{Done}. At the end, again, \textit{Play}, \textit{Submit}, or \textit{Cancel} to start over. 

{\color{red} \faIcon{question}} Should we use primitive shapes for users to choose from? 
The reason for considering this aspect is whether during generation we want to learn to change parameters of a fix set of shapes or generate un-constrained strokes. For the first option, we want users to compose a drawing with primitive shapes, much like using   
In order to learn a more general model, we decided that we want to collect strokes instead of fixed primitive shapes, so we moved onto creating a table that accompanies the drawing board, where the user can choose to annotate each step they draw. 

In Figure \ref{v0.design}. 

\begin{figure*}[ht!]
\begin{subfigure}{\textwidth}
  \centering
  \includegraphics[width=.8\linewidth]{data_collection/version_0_select_primitive.png}
  \caption{Design of main task for third pilot.}
  \label{v0.1}
\end{subfigure}
\newline
\begin{subfigure}{\textwidth}
  \centering
  % include third image
  \includegraphics[width=.8\linewidth]{data_collection/version_0_smiley_flower_with_primitive.png}  
  \caption{Design of main task for final task.}
  \label{v0.2}
\end{subfigure}
\caption{Progress of the design two for the main task in version two.}
\label{v0.design}
\end{figure*}


\section{Version 1}
\subsection{Overview}
When we first started designing our data collection interface, we wanted to collect sketches for prompts similar to the ones used in DALL-E \citep{dallePaper}: creative composition of attributes and objects that are not commonly associated. 
For example, \textit{an evil cup of bubble tea} and \textit{happy moon}. 
In addition to sketches, we also require annotators to decompose their drawings into steps and provide descriptions for each step. We hope to discover combination of simple shapes like examples in Figure \ref{introduction.composition}, and because the imaginative prompts would result in creative sketches, the part descriptions would also be interesting\pdfmarkupcomment[color=yellow]{, like}{want to talk about the YouTube example}.     

We deploy the data collection interface on Amazon Mechanical Turk (AMT), which is a crowdsourcing website that hosts different machine learning annotation tasks. In the remaining text, we use the word \textit{turker} to refer to annotators we recruit on AMT; we will also use the word \textit{HIT}, Human Intelligence Task, to refer to a task hosted on AMT. Please refer to \href{https://www.mturk.com/worker/help#:~:text=A%20Human%20Intelligence%20Task%2C%20or,be%20completed%20by%20Worker%20customers.}{AMT FAQs} for official definition and answers to questions related to AMT. 
 
For each HIT, we need to design: (1) instruction and requirements explaining the dos and don'ts; (2) qualification task to train turkers to provide high-quality annotations; (3) the main interface for data collection. 
% Compared to Version 0, which we only dabbled with \ref{v1_sec_1} in the above list, we went through all three stages for Version 1 and eventually deployed a pilot.
After deploying the pilot, we realized a few major problems with this design. Firstly, due to the subjective nature of sketching, although the sketches are very creative, it was hard to understand how some annotators are illustrating the given prompts. Moreover, turkers are taking more than 30 minutes for each task, and, most importantly, many descriptions do not align with the drawn objects, making the data difficult to use for model learning. For example, in one step, an annotator drew the entire cat face, but they only annotated \textit{big eyes}.      

\subsection{Interface Design}

\subsubsection{Main Task Interface}

\begin{figure*}[!htb]
\begin{subfigure}{\textwidth}
    \centering
    \includegraphics[width=.75\linewidth]{data_collection/v1_empty_table.png}  
\end{subfigure}
\newline
\begin{subfigure}{\textwidth}
    \centering
    % include third image
    \includegraphics[width=.75\linewidth]{data_collection/v1_before_enter_text.png}  
\end{subfigure}
\newline
\begin{subfigure}{\textwidth}
    \centering
    % include third image
    \includegraphics[width=.75\linewidth]{data_collection/v1_after_enter_text.png}  
\end{subfigure}
\caption{A typical annotation process. Top: interface at the start of annotation. Middle: before adding text descriptions for the drawing; red arrow and box show where to click to add text. Bottom: after adding text descriptions for the flower sketch.}
\label{v1.main_task.1}
\end{figure*}

\begin{figure*}[!htb]
\begin{subfigure}{\textwidth}
\centering
\includegraphics[width=.8\linewidth]{data_collection/v1_before_delete.png}  
\end{subfigure}
\newline
\begin{subfigure}{\textwidth}
\centering
\includegraphics[width=.8\linewidth]{data_collection/v1_after_delete.png}  
\end{subfigure}
\caption{Top: the completed annotation for the prompt \textit{Happy Flower}. Red arrows and boxes point to \textit{Delete} buttons that can delete the text annotations along with their drawings. Bottom: after deleting the steps \textit{first round petal} and \textit{smiley mouth}.}
\label{v1.main_task.delete}
\end{figure*}

We illustrate a typical annotation process in Figure \ref{v1.main_task.1}. 
The annotator draws a step on the canvas, enter text description for this step in the \textit{Annotation} column, and hit \textit{Add} to display it as a new row in the annotation table. 
For the annotator's convenience, we include an \textit{Undo} button and a \textit{Clear} button for erasing strokes and clearing the entire canvas. 
If the annotator wants to remove an entire step, including the drawing and the text description, they can use the \textit{Delete} button. An example is shown in Figure \ref{v1.main_task.delete}.
Repeat the drawing-and-adding process until the drawing is done. 
This design encourages turkers to decompose their drawings into semantically meaningful parts.

We encountered some difficulties when implementing the \textit{Delete} button. At the beginning, we treated erasing strokes as drawing the same strokes but in white; however, when strokes overlap each other, overwriting with white strokes breaks other strokes into segments. Therefore, we change the drawing canvas to use layers like Photoshop, so that deleting strokes would be the same as deleting an entire layer, leaving other strokes intact.   

\subsubsection{Instruction and Requirement}

\begin{figure*}[!htb]    
\includegraphics[width=\linewidth]{data_collection/v1_instruction.png} 
\caption{The instruction section used in the prompt-guided sketch text dataset.}
\label{v1.instruction}
\end{figure*}

To ensure that turkers understand the purpose of collecting the dataset, the instruction begins with the motivation behind this project (Figure \ref{v1.instruction}). 
What we struggled the most when drafting the requirements was deciding what a single \textit{step} in sketching was; how do we clearly explain this definition to the turkers? 
% In Version 0, we relied on common geometric shapes to decompose a drawing into a sequence of steps. 
% In Version 1, we considered asking turkers to annotate for each stroke in the drawing, but we quickly ruled out this option since it was time-consuming, and it did not align with how the instructor taught the child in the \textit{How To Draw a Cute Ice-Cream Cone} video. 
% We decided that turkers should annotate for each \textit{object} in the drawing. The ambiguity around the word \textit{object} has posed the biggest challenge in defining a clear set of requirements. 
We considered providing a list of geometric shapes, such as rectangle, triangle, circle, etc., or asking annotations for each stroke, but these options could not reflect how people naturally sketch.

There is a wide spectrum of allowed annotations depending on how people sketch. For example, when drawing for the prompt \textit{Happy Face}, one person might annotate 3 steps: \textit{large u-shaped face}, \textit{round eyes}, \textit{big smiley mouth}. But for someone who likes to draw detailed eyes, they might describe the shape of the eye contour and the length of the eyelashes. 
% So what level of specificity should be allowed?  
The great variation in personal styles makes creative sketches fascinating to study but also challenging to collect a high-quality dataset. 
% The great variation and uncertainty that comes from individuality and personal styles demonstrated through drawings would eventually drive us to not collect drawings and simply ask for text annotations for sketches found in existing datasets.  

We resorted to repeatedly testing the interface with lab mates to refine the requirements. The refinement process is explained in Appendix \ref{appendixDataV1Req}. 
The requirements deployed in the final pilot is shown in Figure \ref{v1.requirement}.
\textcolor{red}{\underline{Bad Example 2}} is shown in Figure \ref{v1.badeg} as an instance of the examples used in the final requirements. To view all the examples, refer to: \url{https://erinzhang1998.github.io/portfolio/amazon_anno}. 


\begin{figure*}[!htb]
\includegraphics[width=\linewidth]{data_collection/v1_requirement.png}  
\caption{Final version of the requirements. The \textcolor{red}{\underline{Bad Example}} links to counter-examples of the requirements, and \textcolor{green}{\underline{Good Example}} links to good examples. When turkers click on the links, they are directed to the examples illustrating the corresponding requirement. This design helps them to understand the requirements better and provides high-quality annotations}
\label{v1.requirement}
\end{figure*}

\begin{figure*}[!htb]
\begin{subfigure}{\textwidth}
\centering
\includegraphics[width=.8\linewidth]{data_collection/v1_badeg_1.png}  
\end{subfigure}
\newline
\begin{subfigure}{\textwidth}
\centering
\includegraphics[width=.8\linewidth]{data_collection/v1_badeg_2.png}  
\end{subfigure}
\newline
\begin{subfigure}{\textwidth}
\centering
\includegraphics[width=.8\linewidth]{data_collection/v1_badeg_3.png}  
\end{subfigure}
\newline
\begin{subfigure}{\textwidth}
\centering
\includegraphics[width=.8\linewidth]{data_collection/v1_badeg_4.png}  
\end{subfigure}
\caption{An example used to explain the requirements to turkers.}
\label{v1.badeg}
\end{figure*}

% Most of the requirements are dedicated to ensure principle \ref{data_design_2} and \ref{data_design_3}. Requirement 1 ensures that no irrelevant sketches and trivial annotations are provided, and we resort to good faith that the annotators would provide a sketch that illustrates the given prompt. As expected, problems related to ambiguous sketches and unaligned text annotations surfaced after the deployment on AMT, eventually resulting in a complete change in format and lead to Version 2.     


\subsubsection{Qualification}

\begin{figure*}[!htb]
\includegraphics[width=\linewidth]{data_collection/v1_qual_header.png}  
\caption{Screenshots of the navigation bar in the qualification test of Version 1.}
\label{v1.qualification.nav}
\end{figure*}

\begin{figure*}[!htb]
\begin{subfigure}{\textwidth}
\centering
\includegraphics[width=.8\linewidth]{data_collection/v1_qual_q9_1.png}  
\end{subfigure}
\newline
\begin{subfigure}{\textwidth}
\centering
\includegraphics[width=.8\linewidth]{data_collection/v1_qual_q9_2.png}  
\end{subfigure}
\caption{Screenshots of question 9 in the qualification test of Version 1.}
\label{v1.qualification.q9}
\end{figure*}

We set up a qualification test on AMT to (1) train turkers to have better understanding of the task and (2) to select turkers who can provide annotations that satisfy all the requirements. Similar to the process of writing the requirements, we went through several rounds of testing with students in the lab to come up with a set of questions that have good correspondence with the requirements. The qualification test starts with the same instruction and requirements that will be used in the final HIT, thus allowing turkers to familiarize themselves with the requirements; moreover, this give them a chance to ask for clarifications before the final HIT. The test leads with a navigation bar (Figure \ref{v1.qualification.nav}) to make it convenient for turkers to switch between questions; originally, we displayed all questions in one page, but some people found it time-consuming to scroll from the later questions back up to the instructions, so we decided to display one question at a time. 
% We have refined the format of the qualification as we gathered feedback from people in the lab; we show a subset of the intermediate versions in Figure.
We show one question from the final qualification in Figure \ref{v1.qualification.q9}. We replicate the exact main task interface in the qualification test, and turkers need to determine whether every step of the mock annotation satisfies all the requirements; we also include hints on which requirement the question is testing for to encourage turkers to revisit the requirements and form better understanding of the task.  
To see the full test, refer to: \url{https://erinzhang1998.github.io/portfolio/amazon_qual}.

% [Figure x5: final qualification test]
%  At first, we asked the annotators to select which steps of the annotations satisfy the requirements (Figure x6.a); in order to use repetition to ensure deep understanding of the requirements, we changed to asking a yes/no question for every step, as shown in Figure x6.b.  



\subsection{Deployment Results}
% In order to determine how feasible the task is, we first deployed a version among lab members, and we obtained 55 drawings along with their annotations. All 55 drawings are shown in Figure v1.results.2. Some examples of step annotations are shown in Figure v1.results.3.  
% [Figure v1.results.2: 55 drawings from lab deployment (see jupyter notebook oct\_28\_trial\_analysis)]
% [Figure v1.results.3: 3 examples of drawings with steps?]
In our first pilot, we used prompts in the forms of \textit{adjective}$\times$\textit{noun}. 
The list of adjectives includes: \textit{happy, sad, surprised, sleepy, love-struck, evil}; the list of nouns includes: 
\textit{person, kid, cat, bear, dog, sheep, jellyfish, cup of bubble tea, apple, burger, sun, moon, star}. 
We want to see what sketches and text descriptions annotators would provide for prompts that ask for imaginative beings not in this world and include novel compositions of unrelated concepts, such as \textit{evil apple} or \textit{love-struck moon}. 
With these creative prompts, we hope to collect data that contain interesting compositions of the same geometric shapes and descriptions across different objects. 
We can then learn models that can, for example, generate circles to be different parts in different objects: eyes, moon, cherries, and angel halo. 

Since we only collected 55 sketches, we were able to manually examine every sketch, and we found many creative sketches [FIGURE]. 
However, one issue was that turkers took a long time, on average 30 minutes, to complete one sketch and provide descriptions. 

The second problem was that it was difficult to understand how some annotators interpreted the prompts through their sketches. [FIGURE]
Indeed, sketching is by its nature very subjective, a common challenge in creative AI.    

The third problem, the most concerning one, was that the part descriptions did not align well with the sketches: some annotators failed to describe every part they drew in a step, or they described parts not in the annotated step. [FIGURE]    


% [Figure v1.results.4: drawings from the amt pilot]
% [Figure v1.results.1: a: oct 28 lab deployment. b: dec 28 amt deployment]
% [Table v1.results.1: comparing the statistics of lab vs. amt deployment]
% Histograms of time each annotator spent on the task is illustrated in Figure v1.results.1. Statistics of the distributions are shown in Table v1.results.1. 
% The discrepancy might be caused by the fact that lab members with their background in computer science have implicit understandings of what kind of quality data are needed to train ML models.   

\section{Version 2}

\subsection{Overview}
In response to the pilot results, we reconsider the data collection pipeline. 
We examined existing sketch datasets to see how their annotations could facilitate our data collection (dataset details are explained in Section \ref{relatedWorkChapter}).    
To shorten the time spent on sketching, we no longer asked annotators to sketch and instead only asked them to describe sketches in the QuickDraw dataset. Although we were no longer able to design text prompts ourselves and collect creative sketches like the ones in the previous pilot, we solved the problem that it was difficult to understand how some sketches illustrate the give prompts. 

The most important objective that this dataset should fulfill is allowing us to study how sketches share similar semantic parts and descriptions that are adapted to be used for different objects. Therefore, we must resolve the issue that some text descriptions we collected in the pilot either did not describe everything drawn in a step or described more than what was drawn.  
% To solve this problem of misaligned text descriptions and drawings, 
Therefore, we used semantic part annotations from the Sketch Perceptual Grouping (SPG) dataset, which provides semantic part labels for each stroke in $20,000$ sketches from the QuickDraw dataset. In this way, annotators did not need to spend time thinking about how to segment sketches into parts themselves. 

Moreover, to help annotators coming up with creative ways to describe the semantic parts, in each task, we presented a pair of sketches with contrasting features, implicitly priming them to describe the visual differences. 
% Between  and SketchSeg, both containing annotaiton for semantically meaningful parts in sketches, 
% SPG annotates for QuickDraw sketches while SketchSeg collects its own sketches. We picked SPG, since it will be easier in the future to extend our datasets given the large QuickDraw reservoir of sketches. 
% Moreover, SketchSeg dataset contains a \textit{fourleg} category that includes many different kinds of animals, such as horse, sheep, and cow, but the QuickDraw categories are more fine-grained, so form a model learning perspective, SPG will also be more generalizable.  


\begin{figure*}[!htb]
\begin{subfigure}{\textwidth}
  \centering
  % include first image
  \includegraphics[width=.8\linewidth]{data_collection/pilot_02_01_annotation_1.png}  
  \caption{Design of main task for first pilot.}
  \label{v2.main_task.1.a}
\end{subfigure}
\newline
\begin{subfigure}{\textwidth}
  \centering
  % include third image
  \includegraphics[width=.8\linewidth]{data_collection/pilot_02_04_annotation.png}  
  \caption{Design of main task for second pilot.}
  \label{v2.main_task.1.b}
\end{subfigure}
\end{figure*}

\begin{figure*}[!htb]
\ContinuedFloat
\begin{subfigure}{\textwidth}
  \centering
  % include third image
  \includegraphics[width=.8\linewidth]{data_collection/pilot_02_04_annotation.png}  
  \caption{Design of main task for third pilot.}
  \label{v2.main_task.1.c}
\end{subfigure}
\newline
\begin{subfigure}{\textwidth}
  \centering
  % include third image
  \includegraphics[width=.8\linewidth]{data_collection/pilot_02_04_annotation.png}  
  \caption{Design of main task for final task.}
  \label{v2.main_task.1.d}
\end{subfigure}
\caption{Progress of the design two for the main task in version two.}
\label{v2.main_task.1}
\end{figure*}

\subsubsection{Main Task Interface}
When collecting the previous prompt-guided dataset, we relied on testing the interface with students in the lab to determine our design, but we observed performance differences between the students and turkers, such as amount of variety in the sketches, amount of time spent on the task, and common confusions related to understanding the requirements. 
Therefore, when collecting the contrasting sketch text dataset, we deployed several pilots on AMT to design the new interface.  
In Figure \ref{v2.main_task.1}, we show how the main task interface progressed from the first pilot to the final version used to collect the entire dataset. 

To better study how similar words are used differently across sketches, we restricted the annotations from whole sentences (Figure \ref{v2.main_task.1.a}) to only adjective phrases (Figure \ref{v2.main_task.1.b}, \ref{v2.main_task.1.c}, \ref{v2.main_task.1.d}). 
We juxtaposed two sketches and highlighted the parts to be annotated in different colors to help annotators notice the contrasting features between the two sketches. 
Moreover, this design expedited the annotation process, since it was easier for people to perform contrasting tasks than to generate descriptions from a single sketch. 

% At the beginning, we stated explicitly that annotators should describe the differences between the sketches (\textit{Describe differences} in \ref{v2.main_task.1.a} and \textit{Compared to Sketch 1/2} in \ref{v2.main_task.1.b}), but we received many annotations that contain comparative and superlatives, so we eventually only have a blank without any introductory phrases to overly emphasize that the goal of the tasks is to create a dataset of contrastive pairs of descriptions, and the juxtaposition is meant only as a mental hint to ease annotation.

\subsubsection{Instruction and Requirement}
At the beginning, the instruction limited the annotators to provide three types of descriptions: shape, size, and position. 
However, in order to collect creative descriptions, we lifted restrictions on the type of words and only required annotators to fill in the blank with adjective phrases. We also provided some examples of adjective phrases in common sentences, unrelated to our task, for annotators to better understand their usage [FIGURE]. 

% In this version, the advantage is that since we have greatly simplify the task to only providing the textual descriptions, the turkers do not have to spend time coming up with drawings for a \textit{adjective}$\times$\textit{noun} prompt, and they do not have to put effort into keeping track of their drawing process to decide how to divide the drawing process into steps and then annotate for each step. 
Since we simplified the HIT from 3 sub-tasks, sketching for the prompt, segmenting the sketches, and describing each step, to only asking for part descriptions, the requirements are much easier to write. 
We did not struggle with explaining what semantic parts in sketches are and how they should be described like previously.  
We relied on the examples in the instruction to give annotators an idea of what descriptions we want. Some examples that we used in the tasks are shown in [FIGURE].
However, the downside for doing so is that the vocabularies used in by the annotators are primed by those in the examples, and we see that annotators would tend to repeat these vocabularies. Therefore, we especially added the requirement that states the annotators are not limited to words used in the examples, and they should use any words that can illustrate the parts well. The full set of requirements used in the final version is shown in Figure \ref{v2.requirement.1}.

\begin{figure*}[!htb]
\begin{subfigure}{0.5\textwidth}
  \centering
  \includegraphics[width=.8\linewidth]{pantanal.jpeg}  
  \caption{Design of main task for first pilot.}
  \label{v2.requirement.examples.1}
\end{subfigure}
\begin{subfigure}{0.5\textwidth}
  \centering
  \includegraphics[width=.8\linewidth]{pantanal.jpeg}  
  \caption{Design of main task for first pilot.}
  \label{v2.requirement.examples.2}
\end{subfigure}
\newline
\begin{subfigure}{0.5\textwidth}
  \centering
  \includegraphics[width=.8\linewidth]{pantanal.jpeg}  
  \caption{Design of main task for second pilot.}
  \label{v2.requirement.examples.3}
\end{subfigure}
\begin{subfigure}{0.5\textwidth}
  \centering
  \includegraphics[width=.8\linewidth]{pantanal.jpeg}  
  \caption{Design of main task for second pilot.}
  \label{v2.requirement.examples.4}
\end{subfigure}
\newline
\begin{subfigure}{0.5\textwidth}
  \centering
  \includegraphics[width=.8\linewidth]{pantanal.jpeg}  
  \caption{Design of main task for second pilot.}
  \label{v2.requirement.examples.5}
\end{subfigure}
\begin{subfigure}{0.5\textwidth}
  \centering
  \includegraphics[width=.8\linewidth]{pantanal.jpeg}  
  \caption{Design of main task for second pilot.}
  \label{v2.requirement.examples.6}
\end{subfigure}
\caption{Progress of the design two for the main task in version two.}
\label{v2.requirement.examples}
\end{figure*}

\begin{figure*}[h]
\includegraphics[width=.8\linewidth]{pantanal.jpeg}  
\caption{The set of requirements used in the final task.}
\label{v2.requirement.1}
\end{figure*}

The requirement that was a bit challenging for people to understand was the one regarding
\textit{Do not use adjectives related to personal opinions, such as random, good, messy, beautiful, and strage, that are hard to achieve consensus if others were to validate your answers.}. Since we hope that the model can get signal from the texts about what kind of figures to draw, words that do not directly convey visual properties of the parts are not helpful. We later changed the wording to \textit{Do not use adjectives that fail to describe specific visual properties of the objects in the sketches}. A slight caveat here is that we actually hope to collect descriptions that describe the emotions expressed in the sketches. We know beforehand that we hope to collect a dataset for the \textit{face} category, so it is quite common for faces to express emotions like happy and sad, and we were slightly worried that some turkers might consider these words as not illustrating enough visual properties about the drawings, since they are quite abstract, at least compared to adjectives like \textit{rectangular} or \textit{wide}. 

\subsubsection{Qualification}
We prepared 10 qualification questions; all are yes/no questions.
We used the qualification test to train turkers to understand the requirements better. 
Each question had a hint that stated which requirement and examples were helpful for solving the question. The purpose of the qualification test was not to trick annotators but to ensure speed and quality of the annotation. 
We show one question from the qualification in Figure \ref{v2.qualification.1}. 
To see the full test, refer to: \url{https://erinzhang1998.github.io/portfolio/v2qual}. 
At the end, we recruited 88 annotators to work on our task. 

\begin{figure*}[!htb]
\centering
\includegraphics[width=\linewidth]{data_collection/version2/v2qualQ3.png}  
\caption{Question 3 from the qualification test used to collect the contrasting sketch text dataset (Section \ref{datav2}). We show a hint at the beginning of each question telling the annotators which requirement this question is testing. In this way, we encourage them to review the requirements so that they have a good understanding of the task and can provide high-quality annotations in the real HIT. The question interface is the same as main task interface that annotators will see when they annotate. The 1-to-1 mock-up helps them to be familiar with the workflow.}
\label{v2.qualification.1}
\end{figure*}

% \begin{table*}[!htb]
% \begin{minipage}[b]{1\textwidth}
% \centering
% \begin{tabular}{l|rrrrrrrrrr}
% \toprule
% Question Number  & 1 & 2 & 2 & 2 & 2 & 2 & 2 & 2 & 2 & 2  \\
% Correct Rate  & 1 & 2 & 2 & 2 & 2 & 2 & 2 & 2 & 2 & 2\\
% \bottomrule
% \end{tabular}
% \caption{Success rate of each question in the qualification test}
% \label{v2.qualification.success_rate}
% \end{minipage}
% \end{table*}
% We released $n$ copies of qualifications, and $n_2$ annotators scored $90$ or higher. The average score for the entire test is $x$, and the rate of correct answer for each question is shown in Table \ref{v2.qualification.success_rate}. Before releasing the qualification, we have tested the test on 



% \subsection{Results}

% \subsubsection{Pilot 1}
% In order to work out the data collection process, we chose the angel category and try to manually examine the sketches and categorize them based on 

% One purpose of the pilot is to estimate the amount of money that we need to spend for each task, and from Table \ref{v2.workertime}, we see that []
% \begin{table*}[!htb]
% \begin{minipage}[b]{1\textwidth}
% \centering
% \begin{tabular}{l|rrrrr}
% \toprule
% ~ & Max. & Min. & Mean & Med. & Std. \\
% \midrule
% Feb 01 Pilot  & 1 & 2 & 2 & 2 & 2   \\
% Feb 04 Pilot  & 1 & 2 & 2 & 2 & 2  \\
% Feb 08 Pilot  & 1 & 2 & 2 & 2 & 2  \\
% Official Collection  & 1 & 2 & 2 & 2 & 2  \\
% \bottomrule
% \end{tabular}
% \caption{Comparing time statistics of pilot task}
% \label{v2.workertime}
% \end{minipage}
% \end{table*}

% For the data collection process, we decide to collect for the face category of the QuickDraw dataset, and the reason for it was mainly to echo the choice of many SOTA generative modeling works that are done on the CelebA dataset. It seems that face generation is quite a starting point for many of the generative modeling work. We have also surveyed some text-to-image synthesis methods that use datasets like (1) CUB dataset (2) MNIST (3) Omniglot. Several sketch datasets include the one from DoodlerGAN and SketchBirds. A lot of the datasets focus on one or two categories, so we decide to do the same to ensure that with our budge, we can collect a dataset that contains enough signal to train a generative ML model. 

% Clustering the faces, we strive to present to the annotators pairs of faces that are distinct as possible in order help them to provide good annotations. It is easier for them to grasp and understand the features of the objects if two sketches are presented in a contrasting way. 

% If we use CLIP to extract the visual features for the entire face sketch.



\section{Dataset Summary}
Our dataset comprises of Quick,Draw! sketches and language descriptions of each semantically meaningful part in the sketch. The dataset contains 2 categories: face and angel, and these categories correspond directly to those in the original Quick!Draw! dataset. The part annotation comes from the SPG dataset \citep{spg_paper}. For the angel category, we annotate for the parts \textit{halo}, \textit{eyes}, \textit{nose}, \textit{mouth}, \textit{body}, \textit{outline of face}, and \textit{wings}. For the face category, we annotate for the parts \textit{eyes}, \textit{nose}, \textit{mouth}, \textit{hair}, \textit{outline of face}. 

\begin{figure*}[h]
\includegraphics[width=\linewidth]{dataset_image/word_freq_face.png}  
\caption{Top 100 most frequent words in the dataset corpus.}
\label{word_freq}
\end{figure*}

\begin{table}[ht!]
\begin{minipage}{1\textwidth}
\begin{center}
{\small
\begin{tabular}{lrrr}
\toprule
~ & Face & Angel \\
\midrule
Number of constrastive pairs & 2515 & 3060 \\
Number of distinct words & 833 & 1107 \\
Number of sketches & 572 & 787 \\
\bottomrule
\end{tabular}}
\caption{Statistics of the dataset by category.}
\label{table:dataset_stats1}
\end{center}
\end{minipage}
\end{table}

\begin{table}[ht!]
\begin{minipage}{1\textwidth}
\begin{center}
{\small
\begin{tabular}{p{9em} | p{1.5em}p{1.5em}p{2em}p{1.5em}p{1.5em} | p{1.5em}p{1.5em}p{1.5em}p{2em}p{1.5em}p{1.5em}p{1.5em} }
\toprule
& \multicolumn{5}{c}{Face} & \multicolumn{7}{c}{Angel}\\
~ & eyes & nose & mouth & hair & face & halo & eyes & nose & mouth & face & body & wings  \\
\midrule
Number of sketches & 
334 & 572 & 572 & 104 & 572 &
558 & 114 & 8 & 80 & 732 & 781 & 779 \\
Number of distinct words & 
228 & 360 & 325 & 152 & 314 & 
365 & 112 & 21 & 88 & 379 & 425 & 534 \\
Number of constrastive pairs &
689 & 401 & 687 & 126 & 612 &
559 & 114 & 8 & 80 & 733 & 785 & 781 \\
\bottomrule
\end{tabular}}
\caption{Statistics of the dataset by sketch parts.}
\label{table:dataset_stats_byparts}
\end{center}
\end{minipage}
\end{table}

In Table \ref{table:dataset_stats_byparts}, we see some statistics about the dataset broken down by sketch parts, while in Table \ref{table:dataset_stats1}, we all list out the same statistics for the entire face and angel category. In general, we observe that compared to previous work that tend to have a fixed list of adjectives for each object parts, the descriptions in our dataset are free-form and non-constrainted. This characteristics is desirable and aligns with our goal to allow robot to collaborate smoothly with humans, since different people would describe the same things in very diverse ways. 