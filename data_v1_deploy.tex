
\subsection{Deployment Results}

% In order to determine how feasible the task is, we first deployed a version among lab members, and we obtained 55 drawings along with their annotations. All 55 drawings are shown in Figure v1.results.2. Some examples of step annotations are shown in Figure v1.results.3.  

% [Figure v1.results.2: 55 drawings from lab deployment (see jupyter notebook oct\_28\_trial\_analysis)]
% [Figure v1.results.3: 3 examples?]
To deploy our first pilot, we need to come up with a set of prompts that are in the forms of \textit{adjective}$\times$\textit{noun}. The list of adjectives includes: 
\textit{happy, sad, surprised, sleepy, lovestruck, evil};
% \textit{happy}, \textit{sad}, \textit{surprised},\textit{sleepy},\textit{lovestruck},\textit{evil}; 
the list of nouns includes: 
\textit{person, kid, cat, bear, dog, sheep, jellyfish, cup of boba, apple, burger, sun, moon, star}.
% \textit{person}, \textit{kid}, \textit{cat}, \textit{bear}, \textit{dog}, \textit{sheep}, \textit{jellyfish}, \textit{cup of boba}, \textit{apple}, \textit{burger}, \textit{sun}, \textit{moon}, \textit{star}. 
We hope to test what drawings and text descriptions annotators would provide for prompts that ask for imaginative beings not in this world, such as \textit{evil apple} or \textit{lovestruck moon}. Our first reason for doing so was that current text-to-image synthesis models, such as DALL-E and GPT-3, can produce creative artwork from abstract prompts that include novel compositions of unrelated concepts; we want to create a dataset that has the capacity to support learning models that can similarly respond to these imaginative prompts through interactive drawing. 
[!] Moreover, if we backtrack to version 0 for a second, the reason why we considered basic geometric shapes was because we are interested in how humans are able to transfer the usage of a circle to different context: a large circle could be a face, an eye, a big piece of cherry, or a moon, so transferring the same visual concept to different sketches. Also there is an aspect of transferring the same language to different context, such as in what ways the adjectives demonstrate the same concept across different object and in what ways they adapt and show different visual qualities when used on different objects.   
% The compositional nature of language allows us to put together concepts to describe both real and imaginary things. We find that DALL·E also has the ability to combine disparate ideas to synthesize objects, some of which are unlikely to exist in the real world. We explore this ability in two instances: transferring qualities from various concepts to animals, and designing products by taking inspiration from unrelated concepts.

[Figure v1.results.4: drawings from the amt pilot]

What surprised us was the amount of time turkers spent on the task. Histograms of time each annotator spent on the task is illustrated in Figure v1.results.1. Statistics of the distributions are shown in Table v1.results.1. The discrepancy might be caused by the fact that lab members with their background in computer science have an implicit understandings of what kind of quality data are needed to train ML models.   

[Figure v1.results.1: a: oct 28 lab deployment. b: dec 28 amt deployment]
[Table v1.results.1: comparing the statistics of lab vs. amt deployment]

Drawing does not illustrate the prompt well. The quality of the drawings are greatly influenced by how well the annotator can understand the prompts. Drawing is by its nature very subjective, so when we were examining through the sketches that we collected, we were not able to understand in what ways some sketches convey the prompts. 
[Figure v1.results.4: some examples of sketches that cannot illustrate the prompt from our perspective]

% In violation of DQ \ref{data_design_2} and \ref{data_design_3}. Another problem was that annotators often fail to describe every part they drew in one step, or the descriptions miss some parts in the step, or the description does not align well with the drawings.   
[Figure v1.results.5: some examples of misaligned descriptions]