We utilize CLIP to gain more insights into our dataset, beyond simple counting statistics. Given the nature of our dataset: a large variety of words but most words have very few occurrences, small number of images, text descriptions are contrastively collected by juxtaposing two images with opposite features, similar descriptions used for different purpose in different context. How does CLIP respond to this dataset, how well does CLIP embeddings align with our intuitions about these tasks? Specifically, for transferring the same word to be used in different context, or usage of words that are not the common meaning of words, how well can CLIP handle them, since it is trained on millions of images? Even though CLIP is not on the sketch domain, CLIP was trained on a large number of images on the internet, and there are GAN methods that have taken advantag eof CLIP embeddings to guide image generation: ClipDraw, StylCLIP, CLIP-NADA.      

\section{Task Definition}
Given two sketches $(s_1,s_2)$ and their part annotations $(t_1,t_2)$, determine which sketch $t_1$ should pair with, similarly for $t_2$. 
During data collection, we implicitly juxtapose two sketches, chosen to be as distinct as possible using some heuristic, either from different clusters or whose cosine distance is large, so the process of annotating the two dissimilar sketches is like the annotators are choosing the pair up one annotation with another. Implicitly, the annotators is pairing $s_1$ with $t_1$ and $s_2$ with $t_2$, so we would regard the ground truth pairing to be $(s_1,t_1)$ and $(s_2,t_2)$. We want to see how CLIP does on this task, if it is the annotator for the task, would it be able to generate the same pairing. Define cosine similarity to be.  

Given $(s_1,s_2)$, we use CLIP image encoder (zero-shot/fine-tuned) $f_v$ to extract visual features for the two sketches,  $f_v(s_1) \in {R}^{512}$, and $f_v(s_2) \in {R}^{512}$. We then use the zero-shot/fine-tuned CLIP text encoder to extract the text features for the part descriptions, namely we fill in the template $t = \texttt{[ADJ] [PART NAME]}$, where $\texttt{[ADJ]}$ is filled with the adjective phrases annotations, and $\texttt{[PART NAME]}$ is the name of the part in the sketches. For angels, $\texttt{[PART NAME]}$ is one of \textit{halo}, \textit{eyes}, \textit{nose}, \textit{mouth}, \textit{body}, \textit{outline of face}, \textit{wings}; for face, $\texttt{[PART NAME]}$ is one of \textit{eyes}, \textit{nose}, \textit{mouth}, \textit{hair}, \textit{outline of face}. After filling in the above template, we obtain the part annotations for the two sketches $t_1,t_2$.  
We obtain embeddings for the part annotations by encoding them through CLIP text encoder $f_t$: $f_t(t_1) \in {R}^{512}$, and $f_t(t_2) \in {R}^{512}$. We then calculate cosine similarity between all four pairs of $(f_v(s_i), f_t(t_j))$, $i,j \in [2]$, where consine similarity between two vectors $u,v$ is defined as $S_c(u,v) = \dfrac{u \cdot v}{\|u\| \|v\|}$. 
Therefore, given that our entire pipeline is $f$, $f(j) \in [2]$ output which of the two sketches $t_j$ will be paired with, and $$f(j) = \max_{i} S_c(f_v(s_i), f_t(t_j)) \hspace{2em} i \in [2]$$.    

\section{Metric}

Given $n$ pairs of two sketches and two part annotations, the same pairs that were provided by the annotators, we calculate an accuracy-like metric:
$$ acc = \frac{\sum_{k=1}^{n} \sum_{j=1}^2 \mathbbm{1}{(f(j) = j)}}{2n} $$

\section{CLIP Finetune}

We load the pretrained model from the \texttt{Python} \texttt{clip} package, specifically the \texttt{ViT-B/32} variant, which uses the Vision Transformer \citep{visiontransformer} as the image encoder; \texttt{B} stands for BERT Base model, and \texttt{32} stands for $32 \times 32$ input patch size. 

\subsection{Image Pre-Processing}
We use the data provided by SPG \citep{spg_paper}, which provides JSON files of the Quick,Draw! sketches in vector format: each sketch is composed of a sequence of $n$ strokes $S_i, i \in [n]$, and $S_i$ is a sequence of vectors $(\delta x,\delta y, p, l)$. $\delta x$ and $\delta y$ are changes in the $x,y$ coordinates with respect to the previous point; for the first point, it is with respect to $(25,25)$. All points are assumed to be drawn on a $256 \times 256$ canvas. $p=1$ if the point is the last point in the current stroke, and $p=0$ otherwise. The SPG dataset provides annotation for semantic segmentation of the sketches, so $l$ is a number representing the semantically meaningful object part.  

% We obtained the rendered sketches by using \texttt{Pycairo}, which is a Python module providing bindings for the cairo graphics library. We use a line width of $5$. After rendering, we manually examined the sketches and filter out face sketches that do not have a pair of eyes, a mouth and the face outline; we also filter out angel sketches that are incomplete or have all the parts merged together, possibly due to collection errors in SPG.   

\subsection{Text Pre-Processing}
We used the \texttt{spacy} package to preprocess the text. \texttt{spacy} provides trained natural language processing pipeline and includes models for, for example, token-to-vector and part-of-speech tagging. We use the \texttt{en\_core\_web\_sm} pipeline and its lemmatizer to reduce words to their basic forms. Moreover, we lower-case all words and remove punctuation, a list of which is provided by \texttt{Python} \texttt{string} package, \texttt{string.punctuation}. We also remove words like \textit{shaped}, \textit{sized}, \textit{and}, \textit{like}, since they act like stop words and do not provide additional visual descriptions of the sketches. Text descriptions are also tokenized by CLIP's tokenizer before passing into CLIP text encoder.     

\section{Loss Function}
During training, for a given batch size $b$, we have $b$ sketch-text pairs, $(s_k,t_k), k\in [b]$. We are essentially using classification over $b$ classes to finetune CLIP, using cross-entropy loss. With clip, we obtain image logits $X_v$ over the text descriptions and text logits $X_t$ over the sketches. The ground-truth, for both image and text, is 
$Y_v = Y_t = \begin{bmatrix}1 & 2 & \cdots & b \end{bmatrix}^T $ 

$$L(X_v, Y_v) = \dfrac{1}{b} \sum_{k=1}^b -\log\frac{\exp{{X_v}_{k,k}}}{ \sum_{c=1}^b \exp{{X_v}_{k,c}} } $$

And similarly defined for $(X_t, Y_t)$,
$$L(X_t, Y_t) = \dfrac{1}{b} \sum_{k=1}^b -\log\frac{\exp{{X_t}_{k,k}}}{ \sum_{c=1}^b \exp{{X_t}_{k,c}} } $$

The final loss is defined as:
$$L = \dfrac{1}{2} (L(X_v, Y_v) + L(X_t, Y_t))$$

\section{Data Augmentation}
As mentioned above, our dataset has a small number of sketches: 572 face sketches and 787 angel sketches.  