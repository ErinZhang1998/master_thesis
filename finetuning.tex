

We fine-tune CLIP with our dataset containing (sketch, part description) pairs. Our sketches come from the QuickDraw dataset \citep{ha2017neural}, and the semantic part annotations come from the SPG dataset \citep{spg_paper}. We collected text descriptions from turkers for every part in face and angel sketches; details of the data collection process is in Section \ref{datav2}, and we summarize the dataset in Section \ref{datasummary}.  
% We will present how we pre-process the image data to fine-tune CLIP (\ref{sketch.preprocess}). Through data collection (Chapter \ref{dataChapter}), we obtained text descriptions of semantic parts provided by SPG, and we present how they are cleaned-up for fine-tuning in (\ref{text.preprocess}).   

\subsubsection*{Sketch Preprocessing} \label{sketch.preprocess}
The QuickDraw sketches are stored in vector format: each sketch is composed of a sequence of $n$ strokes $S_i, i \in [n]$, and each stroke $S_i$ is a sequence of vectors $(\delta x,\delta y, p, l)$. 
$\delta x$ and $\delta y$ are changes in the $x,y$ coordinates with respect to the previous point; for the first point, its coordinate is with respect to the point $(25,25)$. 
All points are assumed to be drawn on a $256 \times 256$ canvas. 
$p=1$ if the point is the last point in the current stroke, and $p=0$ otherwise. 
The SPG dataset provides annotation for semantic segmentation of the sketches, and $l$ is an integer representing which semantic part the current point belongs to. For face sketches, the parts can be \textit{eyes}, \textit{nose}, \textit{mouth}, \textit{hair}, or \textit{outline of face}; for angel sketches, the parts can be \textit{halo}, \textit{eyes}, \textit{nose}, \textit{mouth}, \textit{body}, \textit{outline of face}, or \textit{wings}.
% For angels, $l = 0$ if the point is part of the eyes.   
% $$ for \textit{halo}, \textit{eyes}, \textit{nose}, \textit{mouth}, \textit{body}, \textit{outline of face}, \textit{wings}; for face, $\texttt{[PART NAME]}$ is one of \textit{eyes}, \textit{nose}, \textit{mouth}, \textit{hair}, \textit{outline of face}. 

During training, we render the sketches from the vector format into PNG images using the python package \texttt{cairocffi}. To augment the data, we did: (1) sample stroke width uniformly from the range $[2.5,12.5]$; (2) rotate the sketches randomly by sampling an angle from the range $[-\frac{\pi}{4}, \frac{\pi}{4}]$; (3) scale the sketches with the scale factor sampled from the range $[0.75, 1.25]$; (4) translate the strokes vertically or horizontally with $0.15$ as the maximum absolute fraction translation.   
% 'rotate' : [-1/4*180, 1/4*180],
% 'trans' : [0.15, 0.15],
% 'scale' : [0.75, 1.25],
% 'open_clip' : False,
% 'adamw' : False,
% 'no_image_augment' : False,
% We obtained the rendered sketches by using \texttt{Pycairo}, which is a Python module providing bindings for the cairo graphics library. We use a line width of $5$. After rendering, we manually examined the sketches and filter out face sketches that do not have a pair of eyes, a mouth and the face outline; we also filter out angel sketches that are incomplete or have all the parts merged together, possibly due to collection errors in SPG.  

\subsubsection*{Text Preprocessing} \label{text.preprocess}
We used the \texttt{spacy} package to preprocess the text. \texttt{spacy} provides trained natural language processing pipeline and includes models for, for example, token-to-vector and part-of-speech tagging. We use the \texttt{en\_core\_web\_sm} pipeline and its lemmatizer to reduce words to their basic forms. Moreover, we lower-case all words and remove punctuation, a list of which is provided by \texttt{Python} \texttt{string} package, \texttt{string.punctuation}. We also remove words like \textit{shaped}, \textit{sized}, \textit{and}, \textit{like}, since they act like stop words and do not provide additional visual descriptions of the sketches. Text descriptions are also tokenized by CLIP's tokenizer before passing into CLIP text encoder.     

\subsubsection*{Fine-tuning Objective} \label{finetune.objective}
We used the same loss function and model pipeline as the one used for pre-training CLIP, explained in details in Section \ref{clip.objective}. In this case, in each (image, text) pair, the \textit{image} is a sketch, and the \textit{text} is a description of one of the semantic parts in the sketch. Two examples are shown in Figure \ref{modeling.task.sketches}: through fine-tuning, CLIP should adjust its embedding space so that text feature for \textit{wide triangular body} and image feature for the sketch in Figure \ref{modeling.task.sketches.1} are close in cosine similarity.  

% \subsubsection*{Evaluate CLIP}
% We 

% Given $N$ groups of 2 sketches and 2 part annotations (an example in Figure \ref{modeling.task.sketches}), the same pairs that were provided by the annotators, we calculate an accuracy-like metric:
% $$ acc = \frac{\sum_{k=1}^{n} \sum_{j=1}^2 \mathbbm{1}{(f(j) = j)}}{2n} $$

% % As mentioned above, our dataset has a small number of sketches: 572 face sketches and 787 angel sketches.  
% % How to use CLIP to do this task?
% Given $(s_1,s_2)$, we use CLIP image encoder (zero-shot/fine-tuned) $f_v$ to extract visual features for the two sketches,  $f_v(s_1) \in {R}^{512}$, and $f_v(s_2) \in {R}^{512}$. We then use the zero-shot/fine-tuned CLIP text encoder to extract the text features for the part descriptions, namely we fill in the template $t = \texttt{[ADJ] [PART NAME]}$, where $\texttt{[ADJ]}$ is filled with the adjective phrases annotations, and $\texttt{[PART NAME]}$ is the name of the part in the sketches. For angels, $\texttt{[PART NAME]}$ is one of \textit{halo}, \textit{eyes}, \textit{nose}, \textit{mouth}, \textit{body}, \textit{outline of face}, \textit{wings}; for face, $\texttt{[PART NAME]}$ is one of \textit{eyes}, \textit{nose}, \textit{mouth}, \textit{hair}, \textit{outline of face}. After filling in the above template, we obtain the part annotations for the two sketches $t_1,t_2$.  
% We obtain embeddings for the part annotations by encoding them through CLIP text encoder $f_t$: $f_t(t_1) \in {R}^{512}$, and $f_t(t_2) \in {R}^{512}$. We then calculate cosine similarity between all four pairs of $(f_v(s_i), f_t(t_j))$, $i,j \in [2]$, where consine similarity between two vectors $u,v$ is defined as $S_c(u,v) = \dfrac{u \cdot v}{\|u\| \|v\|}$. 
% \begin{equation}
%     \begin{split}
%         I_1, I_2 \\
%         T_1, T_2 \\
%         S_c(I_i,T_j) & = \dfrac{I_i \cdot T_j}{\|I_i\| \|T_j\|} \\
%         S_c(u,v) & = \dfrac{u \cdot v}{\|u\| \|v\|} \\
%         f(j) & = \max_{i} S_c(f_v(s_i), f_t(t_j)) \hspace{2em} i \in [2]
%     \end{split}
% \end{equation}
% Therefore, given that our entire pipeline is $f$, $f(j) \in [2]$ output which of the two sketches $t_j$ will be paired with, and $$f(j) = \max_{i} S_c(f_v(s_i), f_t(t_j)) \hspace{2em} i \in [2]$$.    

